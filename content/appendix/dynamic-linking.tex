\chapter{Example Code Used in the Dynamic Linking Comparison}\label{appendix:source code run time linking}
This appendix contains two semantically equivalent code examples that differ in the way they resolve Windows API functions. These are used in \autoref{section:comparison load time run time dynamic linking}.

\begin{lstlisting}[caption={Example code that uses load-time dynamically linked function calls.}, label={listing:load-time dynamic linking code}, captionpos=b]
#include <iostream>
#include <Tchar.h>
#include <windows.h>

int main()
{
    const LPCWSTR KEY_NAME = _T(
        "SOFTWARE\\Microsoft\\Windows NT\\CurrentVersion"
    );
    const LPCWSTR VALUE_NAME = _T("ProductName");

    ULONG result;

    HKEY handle;
    WCHAR buffer[128];
    ULONG bufferSize = sizeof(buffer);

    result = RegOpenKeyEx(
        HKEY_LOCAL_MACHINE,
        KEY_NAME,
        NULL,
        KEY_READ | KEY_WOW64_64KEY,
        &handle
    );
    if (result != ERROR_SUCCESS) {
        return result;
    }

    result = RegQueryValueEx(
        handle,
        VALUE_NAME,
        NULL,
        NULL,
        (LPBYTE)buffer,
        &bufferSize
    );
    if (result != ERROR_SUCCESS) {
        return result;
    }

    result = RegCloseKey(handle);
    if (result != ERROR_SUCCESS) {
        return result;
    }

    std::wcout << "'" << buffer << "'" << "\n";

    return 0;
}
\end{lstlisting}

\begin{lstlisting}[caption={Example code that uses run-time dynamically linked function calls.}, label={listing:run-time dynamic linking code}, captionpos=b]
#include <iostream>
#include <Tchar.h>
#include <windows.h>

int main()
{
    typedef LSTATUS(WINAPI *fOpenKey)(
HKEY, LPCWSTR, DWORD, REGSAM, PHKEY
        );
    typedef LSTATUS(WINAPI *fQueryKey)(
        HKEY, LPCWSTR, LPDWORD, LPDWORD, LPBYTE, LPDWORD
        );
    typedef LSTATUS(WINAPI *fCloseKey)(HKEY);

    const LPCWSTR KEY_NAME = _T(
        "SOFTWARE\\Microsoft\\Windows NT\\CurrentVersion"
    );
    const LPCWSTR VALUE_NAME = _T("ProductName");

    ULONG result;

    HKEY handle;
    WCHAR buffer[128];
    ULONG bufferSize = sizeof(buffer);

    HMODULE hLib = LoadLibrary(_T("Advapi32.dll"));
    if (hLib == NULL) {
        return GetLastError();;
    }

    fOpenKey hRegOpenKeyEx = (fOpenKey) GetProcAddress(
        hLib, "RegOpenKeyExW"
    );
    fQueryKey hRegQueryValueEx = (fQueryKey) GetProcAddress(
        hLib, "RegQueryValueExW"
    );
    fCloseKey hRegCloseKey = (fCloseKey) GetProcAddress(
        hLib, "RegCloseKey"
    );
    if (hRegOpenKeyEx == NULL ||
        hRegQueryValueEx == NULL ||
        hRegCloseKey == NULL) {
        return GetLastError();
    }

    result = hRegOpenKeyEx(
        HKEY_LOCAL_MACHINE,
        KEY_NAME,
        NULL,
        KEY_READ | KEY_WOW64_64KEY,
        &handle
    );
    if (result != ERROR_SUCCESS) {
        return result;
    }

    result = hRegQueryValueEx(
        handle,
        VALUE_NAME,
        NULL,
        NULL,
        (LPBYTE)buffer,
        &bufferSize
    );
    if (result != ERROR_SUCCESS) {
        return result;
    }

    result = hRegCloseKey(handle);
    if (result != ERROR_SUCCESS) {
        return result;
    }

    FreeLibrary(hLib);

    std::wcout << "'" << buffer << "'" << "\n";

    return 0;
}
    \end{lstlisting}
