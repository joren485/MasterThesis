\section{Examples of Call Signatures}\label{section:call signatures examples}
Let's look at an example of how to write Call Signatures. In \autoref{listing:sum and add10 function}, we see two simple C functions, \texttt{sum} and \texttt{add10}.

\begin{lstlisting}[label={listing:sum and add10 function}, caption={Two C functions.}, captionpos=b]
int sum(int a, int b){
    return a + b;
}

int add10(int a){
    return sum(10, b);
}
\end{lstlisting}

Given these functions, we have two techniques to add the number 10 to another integer:
\begin{itemize}
    \item \textbf{Add10}: Calling the \texttt{add10} function with an integer as its argument.

    \item \textbf{AddSum}: Calling the \texttt{sum} function with 10 as one of its arguments.
\end{itemize}

Let's say we want to search for these two techniques in a binary. We can use Call Signatures for this.

In \autoref{section:examples searching function calls using Call Signatures} we will expand upon this example, by looking at how to search for function calls using these Call Signatures.

\paragraph{Writing a Call Signature for Add10}

We know that calls to \texttt{add10} have the following specific properties that we can write rules for:
\begin{itemize}
    \item The function name is ``\texttt{add10}''. We need a rule that constrains the function name in the function call to ``\texttt{add10}'':
\begin{lstlisting}[captionpos=b]
- element: "function name"
  equals: "add10"
\end{lstlisting}

    \item The function call has one argument, which we can constrain using a rule for the number of arguments:
\begin{lstlisting}[captionpos=b]
- element: "number of arguments"
  equals: 1
\end{lstlisting}

    \item The argument is an integer. Which we can capture in the following rule:
\begin{lstlisting}[captionpos=b]
- element: "argument"
  argument_index: 0
  type: "number"
\end{lstlisting}

\end{itemize}

Combining all the above rules gives us the Call Signature in \autoref{listing:call signature add10}:
\begin{lstlisting}[label={listing:call signature add10}, caption={A Call Signature for \texttt{add10}.}, captionpos=b]
---

signature:
    technique: "Add10"
    description: >
        This Call Signature can be used to
        search for calls to add10.
    rules:
        - element: "function name"
          equals: "add10"

        - element: "number of arguments"
          equals: 1

        - element: "argument"
          argument_index: 0
          type: "number"
\end{lstlisting}


\paragraph{Writing a Call Signature for AddSum}

Function calls to \texttt{sum} that implement AddSum (i.e. that add 10 to another integer) are similar to those of \texttt{add10}, but in \texttt{sum} either argument can be 10. We use the following rules to constraint each property of such calls:
\begin{itemize}
    \item The function name is ``\texttt{sum}''. We can use the following rule for this:
\begin{lstlisting}[captionpos=b]
- element: "function name"
  equals: "sum"
\end{lstlisting}

    \item The function call has two arguments, which we can constrain using a rule for the number of arguments:
\begin{lstlisting}[captionpos=b]
- element: "number of arguments"
  equals: 2
\end{lstlisting}

    \item The first argument is an integer:
\begin{lstlisting}[captionpos=b]
- element: "argument"
  argument_index: 0
  type: "number"
\end{lstlisting}

\item The second argument is an integer:
\begin{lstlisting}[captionpos=b]
- element: "argument"
  argument_index: 1
  type: "number"
\end{lstlisting}

\item One of the arguments is 10. We can define a rule over all arguments (i.e. that succeeds if at least one argument matches), by using the \texttt{any argument} function call element:
\begin{lstlisting}[captionpos=b]
- element: "any argument"
  equals: 10
\end{lstlisting}

\end{itemize}

Combining all the above rules gives us the Call Signature in \autoref{listing:call signature sum}:
\begin{lstlisting}[label={listing:call signature sum}, caption={A Call Signature for \texttt{sum}.}, captionpos=b]
---

signature:
    technique: "AddSum"
    description: >
        This Call Signature can be used to
        search for calls to sum.
    rules:
        - element: "function name"
          equals: "sum"

        - element: "number of arguments"
          equals: 2

        - element: "argument"
          argument_index: 0
          type: "number"

        - element: "argument"
          argument_index: 1
          type: "number"

        - element: "any argument"
          equals: 10
\end{lstlisting}
