\chapter{Call Signatures}\label{chapter:call signatures}
In this chapter, we present \emph{Call Signatures}: patterns that allow us to search for the function name and arguments of function calls.

In \autoref{chapter:plugin}, we implement an IDA Pro plugin that searches for function calls that match the patterns in Call Signatures. Taken together, the plugin and Call Signatures allow us to search for specific function calls in binaries. By not only constraining the function name but also constraining the arguments of function calls, we can find specific uses of a function (e.g. opening a specific Registry key)

In \autoref{chapter:call signatures for persistence techniques}, we provide several Call Signatures that match function calls that are used to implement persistence techniques (which are discussed in \autoref{chapter:persistence techniques}). In \autoref{chapter:experiments}, we show that it is possible to detect malware capabilities using Call Signatures and our plugin, by detecting persistence techniques in real-world malware samples.

We define Call Signatures and discuss their syntax in detail in \autoref{section:call signatures syntax}. We look at some examples of Call Signatures in \autoref{section:call signatures examples}

\medskip

Call Signatures define constraints on \emph{function call elements}: the function name, the number of arguments, and arguments.

Call Signatures consists of multiple \emph{rules}, expressions that constraint one function call element. For example, if we want to search for the function call \texttt{sum(0, 3)}, we could define the following rules:
\begin{enumerate}
  \item The function name equals ``sum''.
  \item The function takes two arguments.
  \item The first argument equals \texttt{0}.
  \item The second argument equals \texttt{3}.
\end{enumerate}

\medskip

We use the following definitions in this section:
\begin{itemize}
  \item \textbf{Function call element}: One of the following elements that are part of a function call: the function name, the number of arguments and the arguments. See \autoref{section:function calls} for more details.

  \item \textbf{Rule}: A constraint on a function call element. For example, the function name should contain ``sum''.

  \item \textbf{Call Signature}: A combination of rules that together constraint a function call. Note the distinction that Call Signatures define constraints on \emph{function calls} and rules define constraints on \emph{function call elements}.
\end{itemize}

\section{The Syntax of Call Signatures}\label{section:call signatures syntax}
We write Call Signatures in YAML\footnote{\tiny \url{https://yaml.org/}}, a popular format in software development because it is easily readable by both humans and code.

\autoref{listing:example call signature} shows an example of a Call Signature that can be used to search for function calls used in a Registry-based persistence technique. We will discuss the persistence technique in more detail in \autoref{section:registry-based persistence}.

In essence, the example in \autoref{listing:example call signature} describes the following two possible function calls:
\begin{itemize}
  \item \texttt{RegCreateKey(0x80000001, "\path{SOFTWARE\\Microsoft\\Windows\\CurrentVersion\\Run}", ?)}
  \item \texttt{RegOpenKey(0x80000001, "\path{SOFTWARE\\Microsoft\\Windows\\CurrentVersion\\Run}", ?)}
\end{itemize}

\begin{minipage}{0.9\textwidth}
\begin{lstlisting}[label={listing:example call signature}, caption={An example Call Signature. Colors are added for clarity.}, captionpos=b, backgroundcolor={}, escapeinside={\%*}{*}]
---

%*\colorbox{ProcessBlue}{signature}*:
  %*\colorbox{YellowGreen}{technique: "RP0"}*
  %*\colorbox{YellowGreen}{description: >}*
      %*\colorbox{YellowGreen}{This Call Signature can be used to}*
      %*\colorbox{YellowGreen}{search for calls to RegCreateKey or RegOpenKey,}*
      %*\colorbox{YellowGreen}{that are used to implement RP0.}*
  %*\colorbox{Goldenrod}{rules:}*
      %*\colorbox{Goldenrod}{- element: "function name"}*
        %*\colorbox{Goldenrod}{contains\_in:}*
          %*\colorbox{Goldenrod}{- "RegCreateKey"}*
          %*\colorbox{Goldenrod}{- "RegOpenKey"}*

      %*\colorbox{Goldenrod}{- element: "number of arguments"}*
        %*\colorbox{Goldenrod}{equals: 3}*

      %*\colorbox{Goldenrod}{- element: "argument"}*
        %*\colorbox{Goldenrod}{argument\_index: 0}*
        %*\colorbox{Goldenrod}{equals: 0x80000001}*

      %*\colorbox{Goldenrod}{- element: "argument"}*
        %*\colorbox{Goldenrod}{argument\_index: 1}*
        %*\colorbox{Goldenrod}{contains: "\path{SOFTWARE\\Microsoft\\Windows\\CurrentVersion\\Run}"}*
\end{lstlisting}
\end{minipage}

This example in \autoref{listing:example call signature} shows the general structure of a Call Signature:
\begin{itemize}
  \item The three dashes at the top signify the start of a YAML file.

  \item In blue, we see \texttt{signature} All parts of the Call Signature fall under this YAML key.

  \item In green, we see two keys that give some descriptive information about the Call Signature. First, we see a \texttt{technique} key, that allows us to link this Call Signature to a specific technique (i.e. from \autoref{chapter:persistence techniques}). Secondly, we see a \texttt{description} key that allows us to describe what the Call Signature is looking for.

  \item In yellow, we list of rules under the \texttt{rules} key. In \autoref{section:call signature rule structure}, we will discuss what each detail of a rule means.
\end{itemize}

\subsection{Rules}\label{section:call signature rules}
A rule defines a constraint on either the \emph{value} or the \emph{type} of a function call element. It consists of the following:
\begin{itemize}
  \item The function call element that it constrains.
  \item Either one of:
    \begin{itemize}
      \item The type that the function call element should be.

      \item A value and an operator that should be compared to the function call element.
    \end{itemize}
\end{itemize}

\subsubsection{The Syntax of Rules}\label{section:call signature rule structure}
In \autoref{listing:example call signature}, we saw how rules fit into the larger structure of a Call Signature. In this section, we will go into detail about each part of a rule. In \autoref{listing:example call signature rule value}, and \autoref{listing:example call signature rule type} we see two (color-coded) examples of rules. \autoref{listing:example call signature rule value} is a rule that defines a constraint on a value and \autoref{listing:example call signature rule type} is a rule that defines a constraint on a type.

\begin{minipage}[t]{0.40\textwidth}
\begin{lstlisting}[label={listing:example call signature rule value}, caption={An example rule defining a constrains on a value.}, captionpos=b, backgroundcolor={}, escapeinside={\%*}{*}]
- %*\colorbox{YellowGreen}{element: "function name"}*
  %*\colorbox{Goldenrod}{contains}*: %*\colorbox{Orange}{"RegOpenKey"}*
\end{lstlisting}
\end{minipage}\hfill
\begin{minipage}[t]{0.50\textwidth}
\begin{lstlisting}[label={listing:example call signature rule type}, caption={An example rule defining a constrains on a type.}, captionpos=b, backgroundcolor={}, escapeinside={\%*}{*}]
- %*\colorbox{YellowGreen}{element: "number of arguments"}*
  %*\colorbox{ProcessBlue}{type: "number"}*
\end{lstlisting}
\end{minipage}

\begin{itemize}
  \item In both rules, we first see a key value pair with an \texttt{element} key, in green. This key defines which function call element the rule defines a constraint on.

  \item In yellow, we see the operator (i.e. \texttt{contains}) and, in orange, we see the value that the operator should compare with the value of the function call element. This defines that ``RegOpenKey'' should be a substring of the function name (i.e. the function call element that the rule is about)

  \item In blue, we see a key-value pair that defines the type (in this case number) that the function call element (in this case the number of arguments) should have.
\end{itemize}

\subsubsection{Function Call Elements}
Function calls consist of multiple elements (i.e. the function name, the number of arguments, and each argument). To be able to define constraints on all these elements, the \texttt{element} key in a rule can have the following values:
\begin{itemize}
  \item \texttt{function name}: The function name. \autoref{listing:example call signature rule value} shows an example of such a rule.

  \item \texttt{number of arguments}: The number of arguments that are passed to the function. \autoref{listing:example call signature rule type} shows an example of a rule constraining the number of arguments.

  \item \texttt{argument}: A specific argument, passed to the function=. This requires an additional \texttt{argument\_index} YAML key to specify which argument the rule applies to.  The first argument has the index 0 and the arguments are ordered from left to right. For example, \autoref{listing:example call signature rule argument} shows a rule that specifies that the second argument should be a string.

\begin{lstlisting}[label={listing:example call signature rule argument}, caption={An example rule defining a constrains on the second argument.}, captionpos=b]
- element: "argument"
  argument_index: 1
  type: "string"
\end{lstlisting}

  \item \texttt{any argument}: With this function call element, the rule does not apply to a specific argument, but all arguments. In other words, the constraint should be met by at least one argument. The rule in \autoref{listing:example call signature rule value} says that at least one of the arguments passed during a function call should contain the string ``example string''.

\begin{lstlisting}[label={listing:example call signature rule any argument}, caption={An example rule defining a constrains on all arguments.}, captionpos=b]
- element: "any argument"
  contains: "example string"
\end{lstlisting}
\end{itemize}

\subsubsection{Operators}
Constraints on the value of function call elements are defined using operators. An operator defines how the values (the value in the rule and the value of the function call element) are compared. These rules allow us to express constraints such as the function name should equal ``RegCreateKey''.

The following operators are supported:
\begin{itemize}
  \item \texttt{equals}: The two values match if they are exactly the same. \autoref{listing:example call signature rule equals} gives an example of a rule using the \texttt{equals} operator.
\begin{lstlisting}[label={listing:example call signature rule equals}, caption={An example rule using the \texttt{equals} operator.}, captionpos=b]
- element: "function name"
  equals: "sum"
\end{lstlisting}

  \item \texttt{contains}: The values match if the value in the rule is a substring of the function call element. In other words, \texttt{contains: "example string"} matches if the function call element contains the string ``example string''. \autoref{listing:example call signature rule value} shows such a rule.

  This comparison is case-insensitive.

  \item \texttt{in}: Matches, if the value of function call element is an element in a given list. For example, \autoref{listing:example call signature rule in}.
\begin{lstlisting}[label={listing:example call signature rule in}, caption={An example rule using the \texttt{in} operator.}, captionpos=b]
- element: "argument"
  argument_index: 0
  contains_in:
    - 0x80000001
    - 0x80000002
\end{lstlisting}

  \item \texttt{contains\_in}: Matches, if a string in a given list is a substring of the value of function call element. In other words, \texttt{contains: ["string A", "string B"]} matches if the function call element contains either ``string A'' or ``string B''.

  For example, \autoref{listing:example call signature rule in} shows a rule that specifies that at least one argument should contain ``HKCU'' or ``HKEY\_CURRENT\_USER''.
\begin{lstlisting}[label={listing:example call signature rule contains_in}, caption={An example rule using the \texttt{contains\_in} operator.}, captionpos=b]
- element: "any argument"
  contains_in:
    - "HKCU"
    - "HKEY_CURRENT_USER"
\end{lstlisting}
\end{itemize}

\subsubsection{Type}
Constraints on the type of a function call element are defined by the \texttt{type} key. For example, these rules allow us to express constraints as the second argument should be a string.

\autoref{listing:example call signature rule type} shows an example of such a rule. There are multiple available types:
\begin{itemize}
  \item String

  \item Number

  \item Bytes: Sometimes it is useful to match the exact bytes of an argument (and not the string representation of the bytes). For example, in \autoref{section:call signatures dp} and \autoref{section:call signatures tp} we will see that this is useful to match structs with a constant value.

  As YAML does not support bytes natively, we represent bytes as a hexadecimal string. \autoref{listing:example call signature rule bytes} show such a rule.

\begin{lstlisting}[label={listing:example call signature rule bytes}, caption={An example rule specifying a bytes value.}, captionpos=b]
- element: "argument"
  argument_index: 1
  equals: "48 65 6C 6C 6F 20 57 6F 72 6C 64"
  type: "bytes"
\end{lstlisting}
\end{itemize}

Some types are pre-defined. For example, the number of functions is always a number and the function name is always a string.

In some cases, we want to specify a constraint on a value and type at the same time. For example, in \autoref{listing:example call signature rule bytes}, we give a string in the \texttt{equals} operator, but we explicitly specify that it should be interpreted as bytes.

\section{Examples of Call Signatures}\label{section:call signatures examples}
Let's look at an example of how to write Call Signatures. In \autoref{listing:sum and add10 function}, we see two simple C functions, \texttt{sum} and \texttt{add10}.

\begin{lstlisting}[label={listing:sum and add10 function}, caption={Two C functions.}, captionpos=b]
int sum(int a, int b){
    return a + b;
}

int add10(int a){
    return sum(10, b);
}
\end{lstlisting}

Given these functions, we have two techniques to add the number 10 to another integer:
\begin{itemize}
    \item \textbf{Add10}: Calling the \texttt{add10} function with an integer as its argument.

    \item \textbf{AddSum}: Calling the \texttt{sum} function with 10 as one of its arguments.
\end{itemize}

Let's say we want to search for these two techniques in a binary. We can use Call Signatures for this.

In \autoref{section:examples searching function calls using Call Signatures} we will expand upon this example, by looking at how to search for function calls using these Call Signatures.

\paragraph{Writing a Call Signature for Add10}

We know that calls to \texttt{add10} have the following specific properties that we can write rules for:
\begin{itemize}
    \item The function name is ``\texttt{add10}''. We need a rule that constrains the function name in the function call to ``\texttt{add10}'':
\begin{lstlisting}[captionpos=b]
- element: "function name"
  equals: "add10"
\end{lstlisting}

    \item The function call has one argument, which we can constrain using a rule for the number of arguments:
\begin{lstlisting}[captionpos=b]
- element: "number of arguments"
  equals: 1
\end{lstlisting}

    \item The argument is an integer. Which we can capture in the following rule:
\begin{lstlisting}[captionpos=b]
- element: "argument"
  argument_index: 0
  type: "number"
\end{lstlisting}

\end{itemize}

Combining all the above rules gives us the Call Signature in \autoref{listing:call signature add10}:
\begin{lstlisting}[label={listing:call signature add10}, caption={A Call Signature for \texttt{add10}.}, captionpos=b]
---

signature:
    technique: "Add10"
    description: >
        This Call Signature can be used to
        search for calls to add10.
    rules:
        - element: "function name"
          equals: "add10"

        - element: "number of arguments"
          equals: 1

        - element: "argument"
          argument_index: 0
          type: "number"
\end{lstlisting}


\paragraph{Writing a Call Signature for AddSum}

Function calls to \texttt{sum} that implement AddSum (i.e. that add 10 to another integer) are similar to those of \texttt{add10}, but in \texttt{sum} either argument can be 10. We use the following rules to constraint each property of such calls:
\begin{itemize}
    \item The function name is ``\texttt{sum}''. We can use the following rule for this:
\begin{lstlisting}[captionpos=b]
- element: "function name"
  equals: "sum"
\end{lstlisting}

    \item The function call has two arguments, which we can constrain using a rule for the number of arguments:
\begin{lstlisting}[captionpos=b]
- element: "number of arguments"
  equals: 2
\end{lstlisting}

    \item The first argument is an integer:
\begin{lstlisting}[captionpos=b]
- element: "argument"
  argument_index: 0
  type: "number"
\end{lstlisting}

\item The second argument is an integer:
\begin{lstlisting}[captionpos=b]
- element: "argument"
  argument_index: 1
  type: "number"
\end{lstlisting}

\item One of the arguments is 10. We can define a rule over all arguments (i.e. that succeeds if at least one argument matches), by using the \texttt{any argument} function call element:
\begin{lstlisting}[captionpos=b]
- element: "any argument"
  equals: 10
\end{lstlisting}

\end{itemize}

Combining all the above rules gives us the Call Signature in \autoref{listing:call signature sum}:
\begin{lstlisting}[label={listing:call signature sum}, caption={A Call Signature for \texttt{sum}.}, captionpos=b]
---

signature:
    technique: "AddSum"
    description: >
        This Call Signature can be used to
        search for calls to sum.
    rules:
        - element: "function name"
          equals: "sum"

        - element: "number of arguments"
          equals: 2

        - element: "argument"
          argument_index: 0
          type: "number"

        - element: "argument"
          argument_index: 1
          type: "number"

        - element: "any argument"
          equals: 10
\end{lstlisting}

