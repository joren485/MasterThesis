\subsection{False Positives}\label{section:capa false positives}
In the experiment in the previous section, we analyzed the false negatives of CSP and Capa. In this section, we discuss false positives.

Intuitively, Call Signatures and CSP are less prone to false positives than Capa, because Call Signatures operate on a smaller scope (as we discussed in the first difference in \autoref{section:differences call signatures and capa rules}). Capa leaves more room for false positives by looking for specific elements in basic blocks, functions or binaries regardless of how the elements are used and relate to each other.

\medskip

Unfortunately, there is no automated way to know for certain that a sample does not implement a specific technique, which means that finding and analyzing false positives requires a lot of manual work.

\medskip

During our research we noticed eleven samples that were detected (to implement \autoref{DP0}) by Capa, but not by CSP. Manual inspection revealed that these eleven samples do not implement \autoref{DP0}. This shows that our intuition that Call Signatures and CSP are less prone to false positives than Capa is at least anecdotally true.

The decompiled function in \autoref{listing:capa false positive code}\footnote{The hash of the sample is \tiny \texttt{0102777ec0357655c4313419be3a15c4ca17c4f9cb4a440bfb16195239905ade}.} shows us one of these false positives.

\begin{minipage}{0.9\textwidth}
\begin{lstlisting}[label={listing:capa false positive code}, caption={Pseudo code of a call to \texttt{SHGetFolderPath}, decompiled by IDA Pro.}, captionpos=b]
int __usercall sub_1002A2E0@<eax>(int a1@<esi>)
{
  WCHAR pszPath[622]; // [esp+Ch] [ebp-4ECh] BYREF
  int v3; // [esp+4F4h] [ebp-4h]

  *(_DWORD *)(a1 + 20) = 7;
  *(_DWORD *)(a1 + 16) = 0;
  *(_WORD *)a1 = 0;
  v3 = 0;
  memset(pszPath, 0, 0x26Du);
  if ( SHGetFolderPathW(0, 28, 0, 0, pszPath) >= 0 )
    sub_1000BC70(pszPath, a1, wcslen(pszPath));
  return a1;
}
\end{lstlisting}
\end{minipage}

To detect \autoref{DP0}, Capa searches for two elements: a call to \texttt{SHGetFolderPath} and the constant \texttt{7} (representing the user startup directory). We can see that \texttt{SHGetFolderPath} is called in \autoref{listing:capa false positive code} (on line 11), but it is used to resolve the path to the Application Data directory (represented by the constant 28). Capa also finds the constant \texttt{7} in the function (on line 6), but it is unrelated to the call to \texttt{SHGetFolderPath}.

\medskip

CSP does not detect \autoref{DP0} in \autoref{listing:capa false positive code} (and neither does it for the other ten cases), because it looks at a smaller scope (i.e. the function call to \texttt{SHGetFolderPath}) and it does not find a function call that looks like \texttt{SHGetFolderPath(?, 7, ?, ?, ?)}. This shows that tools have a larger scope (like Capa) can be more prone to false positives than tools that have a smaller scope (like Call Signatures and CSP).
