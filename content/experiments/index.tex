\chapter{Experiments \& Analysis}\label{chapter:experiments}
In \autoref{chapter:call signatures}, we presented Call Signatures, expressions to describe function calls of interest. We implemented an IDA Pro plugin, CSP, that searches for function calls using Call Signatures (in \autoref{chapter:plugin}). In \autoref{chapter:call signatures for persistence techniques}, we used Call Signatures to describe persistence techniques (from \autoref{chapter:persistence techniques}).

In this chapter, we will analyze whether CSP detects persistence techniques in known malware. In \autoref{section:real-world experiments}, we create datasets of real-world malware samples that implement each of the four techniques in \autoref{chapter:call signatures for persistence techniques}. We run the CSP on each dataset with the relevant Call Signatures, to see for many samples it detects the persistence techniques.

To see how well CSP and the Call Signatures from \autoref{chapter:call signatures for persistence techniques} perform compared to existing tooling, we run Capa (earlier discussed in \autoref{chapter:related work}) on the same datasets from \autoref{section:real-world experiments}. In \autoref{section:capa comparison experiments}, we analyze how Capa works, how it performs in our experiments, and why.

\section{Detection of Persistence Techniques in Real-World Samples}\label{section:real-world experiments}
To analyze the effectiveness of Call Signatures and CSP, we will run four experiments on real-world malware samples.

In \autoref{chapter:call signatures for persistence techniques}, we wrote Call Signatures for four techniques (one of each category discussed in \autoref{chapter:persistence techniques}): \autoref{RP0}, \autoref{DP0}, \autoref{SP0} and \autoref{TP0}. For each of these techniques, we will create a dataset of samples that implement the technique and try to detect the technique in that dataset using CSP. The goal of these experiments is to see how many true positives and false negatives CSP (using the Call Signatures) finds in a dataset of true positives.

\subsection{The Sources of Malware Samples}\label{section:sources of samples}
To create the datasets, we use the following (publicly available) sources for samples:
\begin{itemize}
    \item APTMalware\footnote{\tiny \url{https://github.com/cyber-research/APTMalware}}: A dataset of state-sponsored malware. This was created by Coen Boot for his master thesis \cite{cboot}.

    \item Capa Test Files\footnote{\tiny \url{https://github.com/mandiant/capa-testfiles}}: Files used by Capa for testing.

    \item theZoo\footnote{\tiny \url{https://github.com/ytisf/theZoo}}: A public repository of live malware samples.
\end{itemize}

Together these datasets contain 3571 unique samples that range from simple banking malware to advanced state-sponsored malware samples.

\subsection{Creating the Datasets}
We use the three sources of samples in \autoref{section:sources of samples} to create four datasets, one for each of the four techniques we will analyze.

To know which samples from \autoref{section:sources of samples} implement each of the techniques, we used commercial dynamic analysis platforms (similar to Hybrid Analysis\footnote{\tiny \url{https://www.hybrid-analysis.com}}, Intezer Analyze\footnote{\tiny \url{https://analyze.intezer.com}}, and the behavioral analysis of VirusTotal\cite{virustotal-behavioural}). These dynamic analysis platforms run malware samples in a (secure and ephemeral) virtual environment. They report on the actions that the samples take (e.g. any IP address the malware tries to connect to) and record the changes in the operating system (e.g. any changed Windows Registry keys).

We used the results of the dynamic analysis of each sample to determine whether it implements \autoref{RP0}, \autoref{DP0}, \autoref{SP0}, or \autoref{TP0}. For example, if the dynamic analysis detected that a sample changed the Registry key \path{HKCU\SOFTWARE\Microsoft\Windows\CurrentVersion\Run}, we know that it implements \autoref{RP0}.

\medskip

It is important to note that dynamic analysis is not a replacement for static analysis. We use it to establish a baseline of true positive samples, but it is not able to find all samples that implement each technique \cite{survey-anti-analysis}.

\medskip

This resulted in the following datasets:
\begin{itemize}
    \item Dataset RP: 162 samples that implement \autoref{RP0}. The hashes can be found in \autoref{appendix:hashes dataset rp}.
    \item Dataset DP: 134 samples that implement \autoref{DP0}. The hashes can be found in \autoref{appendix:hashes dataset dp}.
    \item Dataset SP: 133 samples that implement \autoref{SP0}. The hashes can be found in \autoref{appendix:hashes dataset sp}.
    \item Dataset TP: 28\footnote{We found significantly fewer samples for the TP dataset than for the other dataset. This might be a sign that scheduled task-based persistence techniques are less common than we thought.} samples that implement \autoref{TP0}. The hashes can be found in \autoref{appendix:hashes dataset tp}.
\end{itemize}

\subsection{Analyzing the Datasets}
We analyzed each sample in each dataset with the Call Signatures written for the technique of the dataset. Specifically:
\begin{itemize}
    \item We analyzed each sample in dataset RP with the Call Signatures from \autoref{section:call signatures rp}.
    \item We analyzed each sample in dataset DP with the Call Signatures from \autoref{section:call signatures dp}.
    \item We analyzed each sample in dataset SP with the Call Signatures from \autoref{section:call signatures sp}.
    \item We analyzed each sample in dataset TP with the Call Signatures from \autoref{section:call signatures tp}.
\end{itemize}

Because IDA Pro is designed to analyze a single executable at a time, we wrote a script that can batch process the samples. This script opens each sample in IDA Pro in headless mode (i.e. without a GUI), waits for the IDA Pro analysis to complete, runs the plugin, and stores the IDA Pro output in a log file. After a dataset is fully analyzed, we can analyze the logs of each sample to find how many persistence techniques were detected by CSP.

The average analysis time of a sample is twenty-one seconds. The majority of this time is taken up by the startup and analysis of IDA Pro and decompiling the functions.

\subsection{Discussion of the Results}
\autoref{table:experiment real-world results} shows the results of this analysis. The first column shows the number of samples in each dataset. The second column shows how many samples were detected by CSP to implement the persistence technique in the dataset.

\begin{table}[ht]
    \centering
    \begin{tabular}{l|ll}
        \hline
        Dataset     & Total Samples & Detected by CSP   \\ \hline
        Dataset RP  & 162           & 115 (70.99\%)     \\ \hline
        Dataset DP  & 134           & 134 (100\%)       \\ \hline
        Dataset SP  & 133           & 129 (96.99\%)     \\ \hline
        Dataset TP  &  28           &  19 (67.86\%)     \\ \hline
    \end{tabular}
    \caption{The total number of samples and the number of detected samples by CSP in each dataset.}
    \label{table:experiment real-world results}
\end{table}

As we can see, CSP detected the persistence techniques in most samples. However, for three datasets, CSP did not detect the persistence technique in all samples. This is expected, as the datasets were made using dynamic analysis and the analysis was done using static analysis (we discussed the differences in \autoref{section:malware analysis}).

We can see the advantage of detecting functionality in binaries by looking at function calls in the results of \autoref{DP0} and \autoref{SP0}. Both of these techniques are implemented using specific Windows API function calls (i.e. the functions used to resolve paths and \texttt{CreateService}), making them predictable and hard to hide (because regardless of the obfuscation, the malware will need to make the function call).

On the other hand, the \autoref{RP0} and \autoref{TP0} samples have a significantly lower detection percentage. We assume that this is because both techniques rely more heavily on strings, which are easier to obfuscate. Registry-based persistence techniques always use strings to specify the path of a Registry key and creating a Scheduled Task is more commonly done via the command line (i.e. also using strings), because it is easier to implement than using the Windows API (as we saw in \autoref{section:call signatures tp}). This shows a fundamental weakness in CSP and Call Signatures (and static analysis in general): if data is obfuscated in a binary, it is impossible to know what a binary does, without first deobfuscating the data.

\medskip

We manually analyzed about half of the false negatives (i.e. the samples that were not detected) and we found that the underlying reasons for the false negatives in each dataset fell into two categories\footnote{Analyzing the manual samples took us about three hours.}:
\begin{enumerate}
    \item Limitations of IDA Pro: In some cases, IDA Pro was unable to decompile part of a binary. For example, a common limitation of IDA Pro is that it is unable to decompile exception handling\footnote{\tiny \url{https://www.hex-rays.com/products/decompiler/manual/limit.shtml}}.

    \item Obfuscation by the malware author: As discussed above, malware authors often employ obfuscation to hide data within the binary.
    \begin{itemize}
        \item Some binaries placed data in a special resource section within the binary. When specific data is needed during runtime, it is dynamically resolved in the resource section. This makes it hard to know where specific data is used, without running the binary.

        \item Some binaries encrypt strings by XORing each byte in a string and decrypting the strings at runtime.

        \item Some binaries mix executable code and data throughout the binary, confusing the disassembler.

        \item Some binaries call offsets within functions to confuse the disassembler about what functions they call.
    \end{itemize}
\end{enumerate}

\section{Comparison With Capa}\label{section:capa comparison experiments}
In \autoref{section:real-world experiments}, we showed that CSP is capable of finding persistence techniques. In this section, we will compare CSP to Capa, to get an idea of how well CSP performs compared to existing tooling. As both Capa and CSP are static analysis tools that search for specific elements in a binary based on pre-defined conditions, the tools are easy to compare.

\medskip

It is important to note that we are making this comparison to determine the effectiveness of Call Signatures and CSP, and not to showcase a replacement for Capa.

\subsection{A Short Overview of Capa}
Capa is a static malware analysis tool that detects the capabilities of malware samples. It is not limited to persistence but can detect many capabilities, ranging from simple interactions with the operating system (e.g. writing to a file) to more complex malware capabilities (e.g. communication with a C\&C server). It does not use the disassembler of IDA Pro, but a standalone disassembler called Vivisect\footnote{\tiny \url{https://github.com/vivisect/vivisect}}.

Capa searches for specific elements (e.g. constants or Windows API function calls) in a binary. It pre-defines which elements it is looking for, for a specific capability, in \emph{rules} (to prevent confusion with the rules of Call Signatures, we will refer to the rules in Capa as \emph{Capa rules}). \autoref{listing:capa rule} shows an example of such Capa rule\footnote{\tiny \url{https://github.com/mandiant/capa-rules/blob/master/persistence/startup-folder/get-startup-folder.yml}}.

Each rule consists of meta-information (e.g. the name and the author) and a list of elements that it is looking for. In \autoref{listing:capa rule}, we see an example of a Capa rule.

\begin{minipage}{0.9\textwidth}
\begin{lstlisting}[label={listing:capa rule}, caption={The \texttt{get startup folder} Capa rule.}, captionpos=b]
rule:
  meta:
    name: get startup folder
    namespace: persistence/startup-folder
    author: matthew.williams@mandiant.com
    scope: basic block
    att&ck:
      - Persistence::Boot or Logon Autostart Execution::Registry Run Keys / Startup Folder [T1547.001]
    examples:
      - 07F7846BBCDA782E5639292AD93907EB:0x40121A
  features:
    - and:
      - or:
        - number: 0x07 = CSIDL_STARTUP
        - number: 0x18 = CSIDL_COMMON_STARTUP
      - or:
        - api: shell32.SHGetFolderPath
        - api: shell32.SHGetFolderLocation
        - api: shell32.SHGetSpecialFolderPath
        - api: shell32.SHGetSpecialFolderLocation
\end{lstlisting}
\end{minipage}

Capa searches for the elements in a Capa rule in one of three scopes: a basic block, a function, or a binary. For example, in \autoref{listing:capa rule} all elements need to be present in a basic block. When all elements in a Capa rule are present in the scope of the Capa rule, Capa reports that a binary implements the capability that the rule describes.

In \autoref{listing:capa rule}, we also see that it is possible to use conjunctions and disjunctions that allow users to define more complex expressions of elements (e.g. searching for \texttt{0x07} or \texttt{0x18}). Because of this, Capa rules often define multiple ways to implement a technique (e.g. \autoref{listing:capa rule} defines multiple API functions that resolve paths) or multiple techniques (e.g. \autoref{listing:capa rule} is looking for both \autoref{DP0} and \autoref{DP1}).

\medskip

For a detailed explanation of how Capa and its rules work, please read the documentation provided by Mandiant\footnote{\tiny \url{https://github.com/mandiant/capa-rules/blob/master/doc/format.md}}.

\subsection{The Differences Between Call Signatures \& Capa Rules}\label{section:differences call signatures and capa rules}
Call Signatures and CSP are similar to Capa. They both search for pre-defined patterns in a binary. However, they do have some nuanced, but fundamental differences. In our opinion, the three most important differences are:

\begin{enumerate}
  \item The scope they operate on: Call Signatures and Capa rules are defined over a different scope. Call Signatures are defined over function calls, while Capa rules are defined over basic blocks, functions, or binaries (which can contain multiple function calls). Because Call Signatures use a smaller scope, we can say that Call Signatures are more precise (i.e. more restrictive) than Capa rules.

  We discuss the practical implications of this difference in detail in \autoref{section:capa false positives}.

  \item Being able to detect indirect function calls: CSP is able to find indirect function calls (i.e. calls where the function address is computed at runtime), while Capa is not.

  For example, if a malware sample implements \autoref{DP0} by calling \texttt{SHGetFolderPath} indirectly, the decompiled function call might look \texttt{?(?, 0x07, ?, ?, ?)}. Capa (using the Capa rule in \autoref{listing:capa rule}) will not detect this function call, because it does not see a call to \texttt{SHGetFolderPath}. However, CSP (using a Call Signature from \autoref{section:call signatures dp}) is able to detect this call, because it does not require the function name to be known (discussed in \autoref{section:matching function call call signatures}) and sees that the second argument is \texttt{0x07}.

  \item The taxonomy of techniques: In \autoref{chapter:call signatures for persistence techniques}, we use the clear taxonomy we laid out in \autoref{chapter:persistence techniques} to write a Call Signature for each technique we want to be able to detect.

  Capa does not provide such a taxonomy, making it unclear which Capa rules cover which techniques. This can result in Capa rules describing multiple techniques or techniques not being covered by any Capa rule.

  \autoref{listing:capa rule} is a good example of one Capa rule covering multiple techniques. If the ``get startup folder'' Capa rule results in a match, we know that the malware uses a startup directory to achieve persistence. However, we do not know which directory, because the ``get startup folder'' Capa rule is used to detect both \autoref{DP0} and \autoref{DP1}.
\end{enumerate}

\subsection{Experiments}
To compare the detection performance of CSP (with Call Signatures) and Capa (with Capa rules), we run the same experiments from \autoref{section:real-world experiments} using Capa:

\begin{itemize}
  \item We try to detect \autoref{RP0} in dataset RP, using Capa and the ``persist via Run registry key''\footnote{Some Capa rules describe more than one persistence technique. For example, \autoref{listing:capa rule} describes both \autoref{DP0} and \autoref{DP1}. To address this, we parse the output of Capa to make sure only the relevant elements are present.} Capa rule\footnote{\tiny \url{https://github.com/mandiant/capa-rules/blob/master/persistence/registry/run/persist-via-run-registry-key.yml}}.

  \item We try to detect \autoref{DP0} in dataset DP, using Capa and the ``get startup folder'' Capa rule\footnote{\tiny \url{https://github.com/mandiant/capa-rules/blob/master/persistence/startup-folder/get-startup-folder.yml}}.

  \item We try to detect \autoref{SP0} in dataset SP, using Capa and the ``persist via Windows service'' Capa rule\footnote{\tiny \url{https://github.com/mandiant/capa-rules/blob/master/persistence/service/persist-via-windows-service.yml}}.

  \item We try to detect \autoref{TP0} in dataset TP, using Capa and the ``schedule task via ITaskScheduler''\footnote{\tiny \url{https://github.com/mandiant/capa-rules/blob/master/persistence/scheduled-tasks/schedule-task-via-itaskscheduler.yml}}, ``schedule task via command line''\footnote{\tiny \url{https://github.com/mandiant/capa-rules/blob/master/persistence/scheduled-tasks/schedule-task-via-command-line.yml}}, and ``schedule task via ITaskService''\footnote{\tiny \url{https://github.com/mandiant/capa-rules/blob/master/nursery/schedule-task-via-itaskservice.yml}} Capa rules.
\end{itemize}

It is important to note that we are technically making two comparisons here: how well Call Signatures and Capa rules can describe persistence techniques and how well CSP and Capa can detect persistence techniques using Call Signatures and Capa rules, respectively.

\medskip

\autoref{table:experiment capa comparison} shows the same table as \autoref{table:experiment real-world results} with an additional column that shows the number of detected samples by Capa for each dataset.

\begin{table}[ht]
  \centering
  \begin{tabular}{l|lll}
      \hline
      Dataset     & Total Samples & CSP           & Capa          \\ \hline
      Dataset RP  & 162           & 115 (70.99\%) & 112 (69.14\%) \\ \hline
      Dataset DP  & 134           & 134 (100\%)   & 134 (100\%)   \\ \hline
      Dataset SP  & 133           & 129 (96.99\%) & 121 (90.98\%) \\ \hline
      Dataset TP  & 28            & 19  (67.86\%) & 18  (94.74\%) \\ \hline
  \end{tabular}
  \caption{The number of matches per dataset by CSP compared with the matches by Capa.}
  \label{table:experiment capa comparison}
\end{table}

As we can see, CSP and Capa perform similarly, but Capa performs slightly worse on three of the four datasets. As Call Signatures are more precise than Capa rules (discussed in \autoref{section:differences call signatures and capa rules}), we would expect CSP to detect fewer samples than Capa. It is surprising to see that Capa detects slightly fewer samples than CSP.

\medskip

Further analysis of the samples that were detected by CSP and not by Capa shows that the difference in matches can be explained by the second and third differences in \autoref{section:differences call signatures and capa rules} (i.e. CSP detecting indirect calls and differences in the rules):

\begin{itemize}
  \item The Capa rule for \autoref{RP0} does not fully cover opening Registry keys from the command line (discussed in \autoref{section:background windows registry}).

  \item The Capa rule for \autoref{SP0} does not fully cover creating Windows Services from the command line (discussed in \autoref{section:background windows registry}).

  \item Capa did not detect all calls to \texttt{CreateService} (the API function used in \autoref{SP0}), because the calls were made indirectly.

  \item The Capa rules for \autoref{TP0} do not cover creating a Scheduled Task from the command line via \texttt{at.exe} (discussed in \autoref{section:call signatures tp}).

  \item Capa did not detect all calls to \texttt{CoCreateInstance} (the API function used in \autoref{TP0}), because the calls were made indirectly.
\end{itemize}

Most of these problems can be solved by adding new or modifying existing Capa rules and are not a fundamental limitation of Capa rules or Capa However, this does show the importance of having a well-defined taxonomy of techniques and to not rely on function names.

\subsection{False Positives}\label{section:capa false positives}
In the experiment in the previous section, we analyzed the false negatives of CSP and Capa. In this section, we discuss false positives.

Intuitively, Call Signatures and CSP are less prone to false positives than Capa, because Call Signatures operate on a smaller scope (as we discussed in the first difference in \autoref{section:differences call signatures and capa rules}). Capa leaves more room for false positives by looking for specific elements in basic blocks, functions or binaries regardless of how the elements are used and relate to each other.

\medskip

Unfortunately, there is no automated way to know for certain that a sample does not implement a specific technique, which means that finding and analyzing false positives requires a lot of manual work.

\medskip

During our research we noticed eleven samples that were detected (to implement \autoref{DP0}) by Capa, but not by CSP. Manual inspection revealed that these eleven samples do not implement \autoref{DP0}. This shows that our intuition that Call Signatures and CSP are less prone to false positives than Capa is at least anecdotally true.

The decompiled function in \autoref{listing:capa false positive code}\footnote{The hash of the sample is \tiny \texttt{0102777ec0357655c4313419be3a15c4ca17c4f9cb4a440bfb16195239905ade}.} shows us one of these false positives.

\begin{minipage}{0.9\textwidth}
\begin{lstlisting}[label={listing:capa false positive code}, caption={Pseudo code of a call to \texttt{SHGetFolderPath}, decompiled by IDA Pro.}, captionpos=b]
int __usercall sub_1002A2E0@<eax>(int a1@<esi>)
{
  WCHAR pszPath[622]; // [esp+Ch] [ebp-4ECh] BYREF
  int v3; // [esp+4F4h] [ebp-4h]

  *(_DWORD *)(a1 + 20) = 7;
  *(_DWORD *)(a1 + 16) = 0;
  *(_WORD *)a1 = 0;
  v3 = 0;
  memset(pszPath, 0, 0x26Du);
  if ( SHGetFolderPathW(0, 28, 0, 0, pszPath) >= 0 )
    sub_1000BC70(pszPath, a1, wcslen(pszPath));
  return a1;
}
\end{lstlisting}
\end{minipage}

To detect \autoref{DP0}, Capa searches for two elements: a call to \texttt{SHGetFolderPath} and the constant \texttt{7} (representing the user startup directory). We can see that \texttt{SHGetFolderPath} is called in \autoref{listing:capa false positive code} (on line 11), but it is used to resolve the path to the Application Data directory (represented by the constant 28). Capa also finds the constant \texttt{7} in the function (on line 6), but it is unrelated to the call to \texttt{SHGetFolderPath}.

\medskip

CSP does not detect \autoref{DP0} in \autoref{listing:capa false positive code} (and neither does it for the other ten cases), because it looks at a smaller scope (i.e. the function call to \texttt{SHGetFolderPath}) and it does not find a function call that looks like \texttt{SHGetFolderPath(?, 7, ?, ?, ?)}. This shows that tools have a larger scope (like Capa) can be more prone to false positives than tools that have a smaller scope (like Call Signatures and CSP).


