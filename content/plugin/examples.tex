\section{Examples of Call Signatures Matching}\label{section:examples searching function calls using Call Signatures}
In this section, we will look at an example of matching a function calls and Call Signatures. This is an example of the process laid out in \autoref{section:matching function call call signatures}.

In this example, we will use an example function (\autoref{listing:sum function plugin}) and corresponding Call Signature \autoref{listing:call signature sum plugin} from \autoref{section:call signatures examples}

\begin{lstlisting}[label={listing:sum function plugin}, caption={A C function that adds two integers.}, captionpos=b]
int sum(int a, int b){
    return a + b;
}
\end{lstlisting}

\begin{lstlisting}[label={listing:call signature sum plugin}, caption={A Call Signature for \texttt{sum}.}, captionpos=b]
---

signature:
    technique: "AddSum"
    description: >
        This Call Signature can be used to
        search for calls to sum.
    rules:
        - element: "function name"
          equals: "sum"

        - element: "number of arguments"
          equals: 2

        - element: "argument"
          argument_index: 0
          type: "number"

        - element: "argument"
          argument_index: 1
          type: "number"

        - element: "any argument"
          equals: 10
\end{lstlisting}

\medskip

Let's say that IDA Pro identified the following function calls in a binary. To see if any of them match the Call Signature in \autoref{listing:call signature sum plugin}, we need to iterate through all the rules and see if they succeed when applied to the function call. In the following examples, a ``?'' signifies a function call element with an unknown value.
\begin{itemize}
    \item \texttt{sum(2, 3)}
    \begin{enumerate}
        \item The first rule succeeds because the function name equals ``\texttt{sum}''.

        \item The second rule succeeds because the call takes two arguments.

        \item The third rule succeeds because the first argument is an integer (i.e. a number).

        \item The fourth rule succeeds because the second argument is also an integer.

        \item The fifth rule fails, because no argument is equal to 10.
    \end{enumerate}

    As the fifth rule failed, this call is not a match.

    \item \texttt{?(5, 10)}
    \begin{enumerate}
        \item The first rule succeeds, as the function name is allowed to be unknown. This is explained in detail in \autoref{listing:call signature sum plugin}.

        \item The second rule succeeds because the call takes two arguments.

        \item The third rule succeeds because the first argument is an integer.

        \item The fourth rule succeeds because the second argument is also an integer.

        \item The fifth rule succeeds because an argument (i.e. the second argument) is equal to 10.
    \end{enumerate}

    As all rules succeeded, this call is a match for the Call Signature in \autoref{listing:call signature sum plugin}.

    \item \texttt{sum(10, ?)}
    \begin{enumerate}
        \item The first rule succeeds because the function name equals ``\texttt{sum}''.

        \item The second rule succeeds because the call takes two arguments.

        \item The third rule succeeds because the first argument is an integer.

        \item The fourth rule failed, because the second argument is unknown. Unlike, the function name, rules will always fail on unknown arguments. This is explained in detail in \autoref{listing:call signature sum plugin}.
    \end{enumerate}

    As the fourth rule failed, this call is not considered a match and the fifth rule is not applied.
\end{itemize}
