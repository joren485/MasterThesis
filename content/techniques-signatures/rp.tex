\section{Registry-based Persistence Techniques}\label{section:call signatures rp}
In this section, we will discuss Call Signatures for \autoref{RP0}, a technique that writes to (a subkey of) the Registry key \path{HKCU\SOFTWARE\Microsoft\Windows\CurrentVersion\Run}.

As we saw in \autoref{section:background windows registry}, there are two common ways for applications to modify the Registry, namely the Windows API and the \texttt{reg.exe} command-line application.

\subsection{Modifying the Registry via the API}

To set the value in a Registry key using the Windows API, an application first needs to (create or) open it using one of the following functions (as seen in the examples in \autoref{appendix:source code run time linking}):
\begin{multicols}{2}
  \begin{itemize}
      \item \texttt{RegCreateKey}
      \item \texttt{RegCreateKeyEx}
      \item \texttt{RegOpenKey}
      \item \texttt{RegOpenKeyEx}
  \end{itemize}
\end{multicols}

All of these functions work similarly. The first argument takes a constant that specifies the root key. For example, the constant \texttt{0x80000001} represents \texttt{HKCU}. The second argument specifies the path of the key.

To detect \autoref{RP0} we need to find a call to a function where the first parameter is \texttt{0x80000001} and the second argument contains ``\path{SOFTWARE\Microsoft\Windows\CurrentVersion\Run}''. \autoref{listing:call signature run key windows api} is a Call Signature for such calls.

\begin{lstlisting}[label={listing:call signature run key windows api}, caption={A Call Signature for \autoref{RP0}.}, captionpos=b]
---

signature:
  technique: "RP0"
  description: >
      This Call Signature can be used to
      search for calls to RegCreateKey or RegOpenKey,
      that are used to implement RP0.
  rules:
      - element: "function name"
        contains_in:
          - "RegCreateKey"
          - "RegOpenKey"

      - element: "number of arguments"
        equals: 3

      - element: "argument"
        argument_index: 0
        equals: 0x80000001

      - element: "argument"
        argument_index: 1
        contains: "SOFTWARE\\Microsoft\\Windows\\CurrentVersion\\Run"
\end{lstlisting}

\subsection{Modifying the Registry via Reg.exe}
To modify the Windows Registry using \texttt{reg.exe}, one needs to run \texttt{reg.exe} (or just \texttt{reg}) command with \texttt{add} as the first argument. For example, the Windows command in \autoref{listing:reg.exe example command}.

\begin{lstlisting}[label={listing:reg.exe example command}, caption={A Windows Command that implements \autoref{RP0}.}, captionpos=b]
  reg add "HKEY_CURRENT_USER\Software\Microsoft\Windows\CurrentVersion\Run" /v WindowsUpdate /t REG_SZ /d "C:\Users\John\windowsupdate.exe"
\end{lstlisting}

To match all situations in which such a command is used to implement \autoref{RP0}, we are looking for the strings with the following substrings:
\begin{itemize}
  \item ``reg''
  \item ``add''
  \item ``HKCU'' or ``HKEY\_CURRENT\_USER''
  \item ``\path{SOFTWARE\Microsoft\Windows\CurrentVersion\Run}''
\end{itemize}

As these strings will not always be part of the same argument, we need a Call Signature that has rules for each of them, separately. \autoref{listing:call signature run key reg.exe} does this.

\begin{lstlisting}[label={listing:call signature run key reg.exe}, caption={A Call Signature for \autoref{RP0}.}, captionpos=b]
---

signature:
  technique: "RP0"
  description: >
      This Call Signature can be used to search for
      usage of reg.exe commands, that implement RP0.
  rules:
      - element: "any argument"
        contains: "reg"

      - element: "any argument"
        contains: "add"

      - element: "any argument"
        contains_in:
          - "HKCU"
          - "HKEY_CURRENT_USER"

      - element: "any argument"
        contains: "SOFTWARE\\Microsoft\\Windows\\CurrentVersion\\Run"
\end{lstlisting}
