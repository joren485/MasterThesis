\section{Service-based Persistence Techniques}\label{section:call signatures sp}
In this section, we will discuss Call Signatures for \autoref{SP0}. As we discussed in \autoref{section:scheduled tasks services in windows}, there are multiple ways for an application to create a new service:

\begin{itemize}
    \item Calling the \texttt{CreateService} API function.
    \item Using the \texttt{sc.exe} command-line application.
    \item Writing directly to the Windows Registry.
\end{itemize}

\subsection{Creating Services Using the Windows API}
To create a service that is automatically started using the Windows API, an application needs to call the \texttt{CreateService} function.

The fifth argument of \texttt{CreateService} is a constant that signifies how the new service should be started. As malware wants the service to be started automatically, it uses the \texttt{SERVICE\_AUTO\_START}\footnote{\tiny \url{https://docs.microsoft.com/en-us/windows/win32/api/winsvc/nf-winsvc-createservicew}} constant. This constant has the value 2.

\autoref{listing:call signature CreateService} is a Call Signature that describes calls to \texttt{CreateService} with \texttt{SERVICE\_AUTO\_START} as the fifth argument.

\begin{lstlisting}[label={listing:call signature CreateService}, caption={A Call Signature that matches a call to \texttt{CreateService}.}, captionpos=b]
---

signature:
    technique: "SP0"
    description: >
      This Call Signature can be used to
      search for calls to CreateService.
    rules:
        - element: "function name"
          contains: "CreateService"

        - element: "number of arguments"
          equals: 13

        - element: "argument"
          argument_index: 5
          equals: 0x00000002
\end{lstlisting}

\subsection{Creating Services Using Sc.exe}\label{section:call signature sc.exe}
To create a service using the command line, malware needs to run the Sc.exe command. For example, the command in \autoref{listing:sc.exe command example}

\begin{lstlisting}[label={listing:sc.exe command example}, caption={An example usage of \texttt{sc.exe}.}, captionpos=b]
sc create ExampleService binpath="C:\temp\example.exe" start="auto" obj="LocalSystem"
\end{lstlisting}

To find such commands in a binary, we need a Call Signature that has rules for the two strings that will always be a part of these commands: ``sc'' and ``create''. The Call Signature in \autoref{listing:call signature sc.exe} has these rules.

\begin{lstlisting}[label={listing:call signature sc.exe}, caption={A Call Signature that matches creating a service using the \texttt{sc.exe} application.}, captionpos=b]
---

signature:
    technique: "SP0"
    description: >
      This Call Signature can be used to
      search for usage of sc.exe commands.
    rules:
      - element: "any argument"
        contains: "sc"

      - element: "any argument"
        contains: "create"
\end{lstlisting}

\subsection{Creating Services Using the Registry}
The configuration of services is stored in the Registry in the key \path{HKLM\SYSTEM\CurrentControlSet\Services}. It is possible to write to a subkey of this key directly, to create a new service.

To be able to detect this, we will use the same Call Signatures as in \autoref{section:call signatures rp}, but we will use \path{HKLM\SYSTEM\CurrentControlSet\Services} (instead of \path{HKCU\SOFTWARE\Microsoft\Windows\CurrentVersion\Run}).
