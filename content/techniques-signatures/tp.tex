\section{Scheduled Tasks-based Persistence Techniques}\label{section:call signatures tp}
In this section, we will discuss Call Signatures for \autoref{TP0}. As we discussed in \autoref{section:scheduled tasks services in windows}, there are multiple ways for an application to create a scheduled task:

\begin{itemize}
    \item Directly calling Windows API functions.
    \item Using the \texttt{schtasks.exe} command-line application.
    \item Using the \texttt{at.exe} command-line application.
\end{itemize}

\subsection{Creating Services Using the Windows API}
To create a scheduled task using the Windows API directly, an application needs to interact with the Task Scheduler service. The Windows API provides two ways to achieve this. Both use the \texttt{CoCreateInstance} to create a handler for the Task Scheduler service.

The first method calls \texttt{CoCreateInstance} with the following constants as its arguments. These constants are GUIDs (discussed in \autoref{section:call signatures dp}) The values of the first and fourth argument are:
\begin{itemize}
  \item \texttt{CLSID\_CTaskScheduler}: \texttt{148BD52A-A2AB-11CE-B11F-00AA00530503}
  \item \texttt{IID\_ITaskScheduler}: \texttt{148BD527-A2AB-11CE-B11F-00AA00530503}
\end{itemize}

The second, newer method is similar to the first, but uses different constants:
\begin{itemize}
  \item \texttt{CLSID\_TaskScheduler}: \texttt{0F87369F-A4E5-4CFC-BD3E-73E6154572DD}
  \item \texttt{IID\_ITaskService}: \texttt{2FABA4C7-4DA9-4013-9697-20CC3FD40F85}
\end{itemize}

\autoref{listing:call signature CoCreateInstance} shows a Call Signature that has rules to constrain these constants.

\begin{lstlisting}[label={listing:call signature CoCreateInstance}, caption={A Call Signature that matches a call to \texttt{CoCreateInstance}.}, captionpos=b]
---

signature:
    technique: "TP0"
    description: >
      This Call Signature can be used to
      search for calls to CoCreateInstance.
    rules:
        - element: "function name"
          equals: "CoCreateInstance"

        - element: "number of arguments"
          equals: 5

        - element: "argument"
          argument_index: 0
          in:
            - "2A D5 8B 14 AB A2 CE 11 B1 1F 00 AA 00 53 05 03"
            - "9F 36 87 0F E5 A4 FC 4C BD 3E 73 E6 15 45 72 DD"
          type: "bytes"

        - element: "argument"
          argument_index: 3
          in:
            - "27 D5 8B 14 AB A2 CE 11 B1 1F 00 AA 00 53 05 03"
            - "C7 A4 AB 2F A9 4D 13 40 96 97 20 CC 3F D4 0F 85"
          type: "bytes"
\end{lstlisting}

\subsection{Creating Services Using Schtasks.exe}
It is also possible to create a scheduled task using the command line, with the schtasks.exe application. For example, the command in \autoref{listing:schtasks.exe command example}.

\begin{lstlisting}[label={listing:schtasks.exe command example}, caption={An example \texttt{schtasks.exe} command that runs a fictitious executable when a user is idle for 30 minutes.}, captionpos=b]
schtasks /create /tn WindowsUpdate /tr "c:\windows\real_update.exe /sc onidle /i 30
\end{lstlisting}

Similar to sc.exe in \autoref{section:call signature sc.exe}, we need rules that capture the strings that will always be part of the command: ``schtasks'' and ``/create''. Like the Call Signature in \autoref{listing:call signature schtasks.exe}.

\begin{lstlisting}[label={listing:call signature schtasks.exe}, caption={A Call Signature that matches creating a scheduled task using the \texttt{schtasks.exe} application.}, captionpos=b]
---

signature:
    technique: "TP0"
    description: >
      This Call Signature can be used to
      search for usage of schtasks.exe commands.
    rules:
        - element: "any argument"
          contains: "schtasks"

        - element: "any argument"
          contains: "/create"
\end{lstlisting}

\subsection{Creating Services Using At.exe}
The Call Signature for at.exe is similar to the one for schtasks.exe because the applications themselves are similar. \autoref{listing:at.exe command example} shows an example of an at.exe command.

\begin{lstlisting}[label={listing:at.exe command example}, caption={An example \texttt{at.exe} command that runs a fictitious executable every Monday at 9:00.}, captionpos=b]
at 09:00 /interactive /every:m C:\Windows\System32\WindowsUpdate\reverseshell.exe
\end{lstlisting}

To be able to find function calls that perform some operation on such commands, we need to find all occurrences of ``at'' and ``/every''.

\begin{minipage}{0.9\textwidth}
\begin{lstlisting}[label={listing:call signature at.exe}, caption={A Call Signature that matches creating a scheduled task using the \texttt{at.exe} application.}, captionpos=b]
---

signature:
    technique: "TP0"
    description: >
      This Call Signature can be used to
      search for usage of at.exe commands.
    rules:
        - element: "any argument"
          contains: "at"

        - element: "any argument"
          contains: "/every:"
\end{lstlisting}
\end{minipage}
