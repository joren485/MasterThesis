\chapter{Persistence Techniques}\label{chapter:persistence techniques}
In this chapter, we will look more closely at \emph{persistence}: the ways in which malware maintains access to infected systems. To test Call Signatures and CSP, we will use them to detect persistence in real-world malware samples (in \autoref{chapter:experiments}). Before we can do this, we need to better understand persistence and how it is implemented by malware. We choose to focus on persistence as it is common for malware to implement.

In \autoref{chapter:call signatures for persistence techniques}, we will write Call Signatures that capture the function calls used to implement the techniques discussed in this chapter.

\medskip

Persistence is achieved by using or altering system settings to automatically launch an executable whenever a specific startup event happens (e.g. the system booting or a user logging in). Windows provides multiple ways to automatically run applications on startup. This is a feature of Windows with legitimate use cases (e.g. starting Spotify or antivirus software automatically). These are generally known as \emph{Autostart Extensibility Points} (ASEPs) \cite{asep-charactericstics-detectabilitiy}.

There are hundreds of ASEPs \cite{evading-autoruns}, making it difficult to list all executables that are automatically run. Windows provides a tool, called Autoruns\footnote{\tiny \url{https://docs.microsoft.com/en-us/sysinternals/downloads/autoruns}}, that analyzes many ASEPs on a running Windows system and lists the executables and DLLs that are automatically started, but even this tool is not exhaustive \cite{evading-autoruns}.

However, not every ASEP is useful to malware, as many ASEPs depend on specific applications being installed or specific Windows features being enabled. Malware authors prefer to use ASEPs that are widely applicable (e.g. across all Windows systems) and reliable. Malware often uses multiple ASEPs, to increase their changes of maintaining access.

\medskip

In this chapter, we discuss the ASEPs that are commonly\footnote{According to well-known malware analysis sources such as Practical Malware Analysis \cite{practical-malware-analysis}.} used by malware, which we refer to as persistence techniques. We subdivided these into four categories\footnote{There are more categories of persistence techniques, but these categories cover the commonly used techniques. See {\tiny \url{https://attack.mitre.org/tactics/TA0003/}} for a more complete taxonomy.}: registry-based (\autoref{section:registry-based persistence}), directory-based (\autoref{section:directory-based persistence}), service-based (\autoref{section:service-based persistence}) and scheduled task-based (\autoref{section:scheduled task-based persistence}). We give each technique a unique identifier, to easily reference it later.

Besides a description of how a technique works, we also note under which user the executable is run and whether Administrator privileges are required to implement the technique.

As a point of reference, we provide the relevant MITRE ATT\&CK ID (discussed in \autoref{chapter:background malware}) of each persistence technique. Note that multiple persistence techniques have the same ATT\&CK ID, as they are categorized as the same ATT\&CK technique. This shows that our classification is more fine-grained than the classification of the ATT\&CK framework.

\section{Registry-based Persistence Techniques}\label{section:registry-based persistence}
As discussed in \autoref{section:background windows registry}, the Registry is a database of configuration settings for Windows. Malware commonly use the following techniques to achieve persistence via the Registry.

\begin{enumerate}[label={\textbf{RP\arabic*}:}, ref=RP\arabic*, start=0]
    \item \label{RP0} Executables that are specified in a value in the Registry key \path{HKCU\SOFTWARE\Microsoft\Windows\CurrentVersion\Run} (or a subkey of this key) will run each time the current user logs in \cite{andrea-blog} \cite{run-key}.

    \begin{itemize}[label={}, leftmargin=*]
        \item \textbf{Runs as}: The user that logs in.
        \item \textbf{Administrator rights required}: No
        \item \textbf{ATT\&CK ID}: T1547.001
    \end{itemize}

    \item \label{RP1} The Registry key \path{HKCU\SOFTWARE\Microsoft\Windows\CurrentVersion\RunOnce} works similarly to \autoref{RP0}, but the values are deleted when they are executed \cite{andrea-blog} \cite{run-key}.

    \begin{itemize}[label={}, leftmargin=*]
        \item \textbf{Runs as}: The user that logs in.
        \item \textbf{Administrator rights required}: No
        \item \textbf{ATT\&CK ID}: T1547.001
    \end{itemize}

    \item \label{RP2} \path{HKLM\SOFTWARE\Microsoft\Windows\CurrentVersion\Run} is the system-wide variant of the key in \autoref{RP0}. When a user logs in, all executables defined in values in this key (or a subkey of this key) are run \cite{andrea-blog} \cite{practical-malware-analysis} \cite{run-key}.

    \begin{itemize}[label={}, leftmargin=*]
        \item \textbf{Runs as}: The user that logs in.
        \item \textbf{Administrator rights required}: Yes
        \item \textbf{ATT\&CK ID}: T1547.001
    \end{itemize}

    \item \label{RP3} The Registry key \path{HKLM\SOFTWARE\Microsoft\Windows\CurrentVersion\RunOnce} works similarly to \autoref{RP2}, but the values are deleted when they are executed \cite{andrea-blog} \cite{run-key}.

    \begin{itemize}[label={}, leftmargin=*]
        \item \textbf{Runs as}: The user that logs in.
        \item \textbf{Administrator rights required}: Yes
        \item \textbf{ATT\&CK ID}: T1547.001
    \end{itemize}

    \item \label{RP4} The DLLs specified in the \texttt{AppInit\_DLLs} value of \path{HKLM\SOFTWARE\Microsoft\Windows NT\CurrentVersion\Windows} are loaded into every process that also loads \texttt{User32.dll} (most applications load \texttt{User32.dll}) \cite{andrea-blog} \cite{practical-malware-analysis}.

    \begin{itemize}[label={}, leftmargin=*]
        \item \textbf{Runs as}: The process that the DLL is loaded into.
        \item \textbf{Administrator rights required}: Yes
        \item \textbf{ATT\&CK ID}: T1546.010
    \end{itemize}

    \item \label{RP5} Windows runs \texttt{userinit.exe} when a user logs in to initialize the user session. This executable is specified in the \texttt{Userinit} value in \path{HKLM\SOFTWARE\Microsoft\Windows NT\CurrentVersion\Winlogon}. The \texttt{Userinit} value can be changed to a comma-separated list of executables that will all be run when a user logs in \cite{andrea-blog} \cite{winlogon-persistence}.

    \begin{itemize}[label={}, leftmargin=*]
        \item \textbf{Runs as}: The user that logs in.
        \item \textbf{Administrator rights required}: Yes
        \item \textbf{ATT\&CK ID}: T1547.004
    \end{itemize}

    \item \label{RP6} Windows runs \texttt{explorer.exe} when a user logs in to set up the graphical user interface. This executable is specified in the \texttt{Shell} value in \path{HKLM\SOFTWARE\Microsoft\Windows NT\CurrentVersion\Winlogon}. Like \autoref{RP5}, this value can be changed into a comma-separated list of executables \cite{andrea-blog} \cite{winlogon-persistence}.

    \begin{itemize}[label={}, leftmargin=*]
        \item \textbf{Runs as}: The current user.
        \item \textbf{Administrator rights required}: Yes
        \item \textbf{ATT\&CK ID}: T1547.004
    \end{itemize}
\end{enumerate}

\section{Directory-based Persistence Techniques}\label{section:directory-based persistence}
Windows provides special startup directories. Executables (or shortcuts to executables) in these directories are run when a user logs in. Malware can simply place its executable file in such a directory, and it will be started every time a user logs in.

\begin{enumerate}[label={\textbf{DP\arabic*}:}, ref=DP\arabic*, start=0]
    \item \label{DP0} Windows provides a startup directory for every user \cite{f-secure-persistence}.

    The default path of this directory is \path{C:\Users\<user>\AppData\Roaming\Microsoft\Windows\Start Menu\Programs\Startup}.

    \begin{itemize}[label={}, leftmargin=*]
        \item \textbf{Runs as}: The user that logs in.
        \item \textbf{Administrator rights required}: No
        \item \textbf{ATT\&CK ID}: T1547.001
    \end{itemize}

    \item \label{DP1} Windows also provides a system-wide startup directory. Files in these directories are run when any user logs in \cite{f-secure-persistence}.

    The default path of this directory is \path{C:\ProgramData\Microsoft\Windows\Start Menu\Programs\Startup}.

    \begin{itemize}[label={}, leftmargin=*]
        \item \textbf{Runs as}: The user that logs in.
        \item \textbf{Administrator rights required}: Yes
        \item \textbf{ATT\&CK ID}: T1547.001
    \end{itemize}
\end{enumerate}

\section{Service-based Persistence Techniques}\label{section:service-based persistence}
As discussed in \autoref{section:scheduled tasks services in windows}, a Windows service is a background process that is started at boot time. Malware uses this to ensure that it is started as a background task.

\begin{enumerate}[label={\textbf{SP\arabic*}:}, ref=SP\arabic*, start=0]
    \item \label{SP0} Malware can create a service that starts its executable in the background on boot. As this service can run as the \texttt{LocalSystem} account, this would immediately give the malware full administrative privileges.

    \begin{itemize}[label={}, leftmargin=*]
        \item \textbf{Runs as}: Services can run as any user account, \texttt{LocalService}, \texttt{NetworkService}, or \texttt{LocalSystem}. See \autoref{section:scheduled tasks services in windows} for more detailed information on these accounts.

        \item \textbf{Administrator rights required}: Yes

        \item \textbf{ATT\&CK ID}: T1043.003
    \end{itemize}
\end{enumerate}

As services are configured in the Windows Registry, there is some overlap between \autoref{SP0} and \autoref{section:registry-based persistence}. However, as the goal is to start a service, we felt it best to add these into their own category.

\section{Scheduled Task-based Persistence Techniques}\label{section:scheduled task-based persistence}
Scheduled Tasks are a way to run applications at pre-defined events or times in Windows (discussed in \autoref{section:scheduled tasks services in windows}). Like services, malware can use them for persistence.

\begin{enumerate}[label={\textbf{TP\arabic*}:}, ref=TP\arabic*, start=0]
    \item \label{TP0} Malware can create a scheduled task that runs on boot or a schedule (e.g. every hour) to ensure persistence. As this task can run as the \texttt{LocalSystem} account, this would immediately give the malware full administrative privileges.

    \begin{itemize}[label={}, leftmargin=*]
        \item \textbf{Runs as}: Scheduled Tasks can run as a specific user, \texttt{LocalService}, \texttt{NetworkService}, or \texttt{LocalSystem}. See \autoref{section:scheduled tasks services in windows} for more detailed information on these accounts.

        \item \textbf{Administrator rights required}: Yes

        \item \textbf{ATT\&CK ID}: T1053.005
    \end{itemize}
\end{enumerate}

