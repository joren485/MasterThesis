\section{Function Calls}\label{section:function calls}
Functions (sometimes called \emph{subroutines}) are an important part of modern programming languages. They allow splitting up code into reusable blocks. Functions can take inputs, called \emph{arguments}, and can return an output value. Executing the code in a function is done by \emph{calling} the function.

In C++ (and most other programming languages), function calls are expressed using the following format: \texttt{function\_name(arg 0, arg 1, ..., arg n)}. Function calls consist of the following \emph{function call elements}:
\begin{enumerate}
    \item The function name.
    \item The number of arguments.
    \item The value and type of each argument.
\end{enumerate}

For example, the function call \texttt{sum(1, 2)} to the function in \autoref{listing:sum function} tells us that:
\begin{enumerate}
    \item The function name is ``\texttt{sum}''.
    \item The function takes two arguments.
    \item The first argument is an \texttt{int}, \texttt{1}.
    \item The second argument is also an \texttt{int}, \texttt{2}.
\end{enumerate}

\begin{lstlisting}[label={listing:sum function}, caption={A C function that adds to integers together.}, captionpos=b]
int sum(int a, int b){
    return a + b;
}
\end{lstlisting}

\subsection{Calling Conventions}\label{section:calling conventions}
Functions are supported in assembly, through the \texttt{call} and \texttt{ret} instructions (discussed in \autoref{section:control flow instructions}). However, assembly does not explicitly support passing arguments to functions. Instead, arguments have to be stored in memory or registers before a function call such that the function will be able to access them. A \emph{calling convention} defines how arguments are passed to a function, on the assembly level. More specifically, a calling convention defines:
\begin{itemize}
    \item How arguments are passed to the callee. This can be on the stack, in registers, or both.
    \item The order in which arguments are passed to the callee.
    \item Whether the callee or caller cleans up the stack after the callee is finished.
    \item How return values are passed to the caller.
\end{itemize}

Calling conventions are specific to a function, which means that multiple calling conventions can be used throughout a single executable.

As we will see in \autoref{section:decompiling function calls}, calling conventions are important when analyzing function calls.

\medskip

There are many calling conventions in x86. We will discuss the three most prominent: cdecl, stdcall, and fastcall.

\subsubsection{Cdecl}\label{section:cdecl}
The cdecl\footnote{\tiny \url{https://docs.microsoft.com/en-us/cpp/cpp/cdecl?view=msvc-170}} convention is the default calling convention used by C. Arguments are pushed to the stack, right to left (i.e. the first argument is pushed to the stack last). The caller cleans up the stack. The return value is passed via \texttt{eax}.

The example C code and corresponding assembly\footnote{Compiled without any optimizations.} in \autoref{listing:sum cdecl} and \autoref{listing:main cdecl} shows cdecl in practice.

\begin{enumerate}
    \item \texttt{main} pushes the arguments to the stack (lines 7 and 9).
    \item \texttt{main} calls \texttt{sum} (line 10).
    \item \texttt{sum} runs and saves its result in \texttt{eax} (line 4).
    \item \texttt{main} cleans up the stack after the call (line 11), by lowering the top of the stack by 8 (the size of two integers).
\end{enumerate}

\begin{lstlisting}[label={listing:sum cdecl}, caption={The C code and assembly of a function that uses cdecl.}, captionpos=b]
int sum(int a, int b) { push    ebp
    return a + b;       mov     ebp, esp
}                       mov     eax, [ebp + 8]
                        add     eax, [ebp + 12]
                        pop     ebp
                        retn
\end{lstlisting}

\begin{lstlisting}[label={listing:main cdecl}, caption={The C code and assembly of a function calling the function from \autoref{listing:sum cdecl} using cdecl.}, captionpos=b]
int main() {            push    ebp
    int x = 1;          mov     ebp, esp
    int y = 2;          sub     esp, 8
    return sum(x, y);   mov     [ebp - 8], 1
}                       mov     [ebp - 4], 2
                        mov     eax, [ebp - 4]
                        push    eax
                        mov     ecx, [ebp - 8]
                        push    ecx
                        call    sum
                        add     esp, 8
                        mov     esp, ebp
                        pop     ebp
                        retn
\end{lstlisting}

\subsubsection{Stdcall}
The stdcall\footnote{\tiny \url{https://docs.microsoft.com/en-us/cpp/cpp/stdcall?view=msvc-170}} convention is used to call Win32 API functions. It is similar to cdecl, however, in stdcall, the callee cleans up the stack after it is done.

\subsubsection{Fastcall}
Fastcall\footnote{\tiny \url{https://docs.microsoft.com/en-us/cpp/cpp/fastcall?view=msvc-170}} is a calling convention designed by Microsoft. It is meant to offer a performance boost over cdecl and stdcall by passing some arguments via registers. The first two arguments that fit in a register are passed in \texttt{ecx} and \texttt{edx}, respectively. All other arguments are pushed to the stack right to left. In fastcall, the callee is responsible for cleaning up the stack after it is done.

In \autoref{listing:sum fastcall}, and \autoref{listing:main fastcall} we see C code and assembly similar to the code in \autoref{section:cdecl}. \texttt{sum} in \autoref{listing:sum fastcall} uses fastcall\footnote{This can be explicitly specified in C code by adding the fastcall keyword to the function declaration. For example, \texttt{int \_\_fastcall sum(int a, int b, int c)}.}.

\begin{enumerate}
    \item \texttt{main} stores the first two arguments to \texttt{edx} and \texttt{ecx} (lines 9 and 10).
    \item \texttt{main} pushes the third argument to the stack (line 8).
    \item \texttt{main} calls \texttt{sum} (line 11).
    \item \texttt{sum} runs and saves its result in \texttt{eax} (line 8).
    \item \texttt{sum} cleans up the stack (line 11) by lowering the top of the stack by 4 (the size of one integer). This implicitly happens in the \texttt{retn 4} instruction.
\end{enumerate}

\begin{lstlisting}[label={listing:sum fastcall}, caption={The C code and assembly of a function that uses fastcall.}, captionpos=b]
int sum(int a,int b,int c){ push    ebp
    return a + b + c;       mov     ebp, esp
}                           sub     esp, 8
                            mov     [ebp - 8], edx
                            mov     [ebp - 4], ecx
                            mov     eax, [ebp - 4]
                            add     eax, [ebp + 8]
                            add     eax, [ebp + 8]
                            mov     esp, ebp
                            pop     ebp
                            retn    4
\end{lstlisting}

\begin{lstlisting}[label={listing:main fastcall}, caption={The C code and assembly of a function calling the function from \autoref{listing:sum fastcall} using fastcall.}, captionpos=b]
int main() {                push    ebp
    int x = 1;              mov     ebp, esp
    int y = 2;              sub     esp, 12
    int z = 3;              mov     [ebp - 12], 1
    return sum(x, y, z);    mov     [ebp - 8], 2
}                           mov     [ebp - 4], 3
                            mov     eax, [ebp - 4]
                            push    eax
                            mov     edx, [ebp - 8]
                            mov     ecx, [ebp - 12]
                            call    sum
                            mov     esp, ebp
                            pop     ebp
                            retn
\end{lstlisting}

\subsection{Decompiling Function Calls}\label{section:decompiling function calls}
Compiling is the process of translating source code into executable machine code. \emph{Decompiling} is the reverse process: translating machine code back into source code. Decompilers are important in reverse engineering software (especially malware analysis), because, if done well, they allow a far better understanding of the functionality of an executable binary. As information about the source code is lost during compilation, it is often not possible to perfectly translate a binary into its source code. Decompilation often relies on heuristics to structure machine code into source code.

\medskip

In \autoref{chapter:plugin}, we will use a decompiler (from IDA Pro) to decompile function calls. As function calls are not standardized in assembly, decompiling function calls is a complex process, which is also heavily dependent on heuristics. When we find a \texttt{call} instruction that we want to decompile, there are two steps to reconstructing the function call:
\begin{enumerate}
    \item Determine the calling convention.

    As we discussed in \autoref{section:calling conventions}, there are various calling conventions that are commonly used (e.g. cdecl is the default calling convention in C++). However, compilers have total freedom in deciding how arguments are passed to a function in assembly and compilers for different programming languages often use custom calling conventions.

    The two important characteristics that differentiate calling conventions from one another are how arguments are passed and how the memory is cleaned up after the function call has finished. If we can discover how these two characteristics are implemented for a function call, we determine what calling convention is used.

    \item If we know the calling convention, we know how arguments are passed to the callee and in what order they are passed.

    We can use this information to backtrack from the \texttt{call} instruction and detect instructions that place data in the locations (i.e. registers or the stack) that are used to pass the arguments.

    For example, if we know that an argument is passed via the \texttt{ecx} register, we will look for instructions that store data in the \texttt{ecx} register.

    However, as arguments can be computed at runtime and compilers optimize memory usage, it is often impossible to reconstruct an argument.
\end{enumerate}

\subsubsection{An Example of Decompiling a Function Call}
Let's look at an example of decompiling a function call. \autoref{listing:call address sum} shows a simple example of a function call (to call the \texttt{sum} function in \autoref{listing:sum function}).

\begin{minipage}{0.9\textwidth}
\begin{lstlisting}[label={listing:call address sum}, caption={The C and assembly code of a function that calls \autoref{listing:sum function} (using cdecl).}, captionpos=b]
int main() {            push    ebp
    int x = 1;          mov     ebp, esp
    int y = 2;          sub     esp, 8
    return sum(x, y);   mov     [ebp - 8], 1
}                       mov     [ebp - 4], 2
                        mov     eax, [ebp - 4]
                        push    eax
                        mov     ecx, [ebp - 8]
                        push    ecx
                        call    <address>
                        add     esp, 8
                        mov     esp, ebp
                        pop     ebp
                        retn
\end{lstlisting}
\end{minipage}

\medskip

The first step is to determine the calling convention. We do this by answering how arguments are passed and whether the callee or the caller cleans up the stack.

How are arguments passed? In the instructions right before the \texttt{call instruction}, from lines 4 to 9, we see that two integers (\texttt{1} on line 4 and \texttt{2} on line 5) are pushed to the stack. This is a good indication that either cdecl or stdcall are used.

Does the callee or the caller clean up the stack? Right after the function call, on line 11, we see that the stack pointer is reduced by 8 bytes (the size of two integers). This tells us that the \texttt{main} function (i.e. the caller) cleans up the stack. A common calling convention that requires the caller to clean up the stack after a function call is cdecl.

By answering these two questions, we have determined that the calling convention cdecl is most likely used.

\medskip

The second step is to find the arguments that are passed to the function. As we noted earlier, we see that two integers are pushed to the stack (in the instructions on lines 4 to 9). Backtracking from the call signature, we first encounter 2 being pushed on the stack and after that we encounter 1 being pushed on the stack.

This tells us what the function call looks like \texttt{?(1, 2)}. The only part that we do not know is the function name. And because compilers often do not use function names from source code, we will not be able to reconstruct the function name.

If a callee is not part of the binary itself, but available in a library, the name might be available. We will discuss this in more detail in \autoref{section:dlls}.

